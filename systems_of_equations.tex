% Options for packages loaded elsewhere
\PassOptionsToPackage{unicode}{hyperref}
\PassOptionsToPackage{hyphens}{url}
\PassOptionsToPackage{dvipsnames,svgnames,x11names}{xcolor}
%
\documentclass[
  letterpaper,
  DIV=11,
  numbers=noendperiod]{scrartcl}

\usepackage{amsmath,amssymb}
\usepackage{iftex}
\ifPDFTeX
  \usepackage[T1]{fontenc}
  \usepackage[utf8]{inputenc}
  \usepackage{textcomp} % provide euro and other symbols
\else % if luatex or xetex
  \usepackage{unicode-math}
  \defaultfontfeatures{Scale=MatchLowercase}
  \defaultfontfeatures[\rmfamily]{Ligatures=TeX,Scale=1}
\fi
\usepackage{lmodern}
\ifPDFTeX\else  
    % xetex/luatex font selection
\fi
% Use upquote if available, for straight quotes in verbatim environments
\IfFileExists{upquote.sty}{\usepackage{upquote}}{}
\IfFileExists{microtype.sty}{% use microtype if available
  \usepackage[]{microtype}
  \UseMicrotypeSet[protrusion]{basicmath} % disable protrusion for tt fonts
}{}
\makeatletter
\@ifundefined{KOMAClassName}{% if non-KOMA class
  \IfFileExists{parskip.sty}{%
    \usepackage{parskip}
  }{% else
    \setlength{\parindent}{0pt}
    \setlength{\parskip}{6pt plus 2pt minus 1pt}}
}{% if KOMA class
  \KOMAoptions{parskip=half}}
\makeatother
\usepackage{xcolor}
\setlength{\emergencystretch}{3em} % prevent overfull lines
\setcounter{secnumdepth}{5}
% Make \paragraph and \subparagraph free-standing
\makeatletter
\ifx\paragraph\undefined\else
  \let\oldparagraph\paragraph
  \renewcommand{\paragraph}{
    \@ifstar
      \xxxParagraphStar
      \xxxParagraphNoStar
  }
  \newcommand{\xxxParagraphStar}[1]{\oldparagraph*{#1}\mbox{}}
  \newcommand{\xxxParagraphNoStar}[1]{\oldparagraph{#1}\mbox{}}
\fi
\ifx\subparagraph\undefined\else
  \let\oldsubparagraph\subparagraph
  \renewcommand{\subparagraph}{
    \@ifstar
      \xxxSubParagraphStar
      \xxxSubParagraphNoStar
  }
  \newcommand{\xxxSubParagraphStar}[1]{\oldsubparagraph*{#1}\mbox{}}
  \newcommand{\xxxSubParagraphNoStar}[1]{\oldsubparagraph{#1}\mbox{}}
\fi
\makeatother


\providecommand{\tightlist}{%
  \setlength{\itemsep}{0pt}\setlength{\parskip}{0pt}}\usepackage{longtable,booktabs,array}
\usepackage{calc} % for calculating minipage widths
% Correct order of tables after \paragraph or \subparagraph
\usepackage{etoolbox}
\makeatletter
\patchcmd\longtable{\par}{\if@noskipsec\mbox{}\fi\par}{}{}
\makeatother
% Allow footnotes in longtable head/foot
\IfFileExists{footnotehyper.sty}{\usepackage{footnotehyper}}{\usepackage{footnote}}
\makesavenoteenv{longtable}
\usepackage{graphicx}
\makeatletter
\def\maxwidth{\ifdim\Gin@nat@width>\linewidth\linewidth\else\Gin@nat@width\fi}
\def\maxheight{\ifdim\Gin@nat@height>\textheight\textheight\else\Gin@nat@height\fi}
\makeatother
% Scale images if necessary, so that they will not overflow the page
% margins by default, and it is still possible to overwrite the defaults
% using explicit options in \includegraphics[width, height, ...]{}
\setkeys{Gin}{width=\maxwidth,height=\maxheight,keepaspectratio}
% Set default figure placement to htbp
\makeatletter
\def\fps@figure{htbp}
\makeatother

\KOMAoption{captions}{tableheading}
\makeatletter
\@ifpackageloaded{tcolorbox}{}{\usepackage[skins,breakable]{tcolorbox}}
\@ifpackageloaded{fontawesome5}{}{\usepackage{fontawesome5}}
\definecolor{quarto-callout-color}{HTML}{909090}
\definecolor{quarto-callout-note-color}{HTML}{0758E5}
\definecolor{quarto-callout-important-color}{HTML}{CC1914}
\definecolor{quarto-callout-warning-color}{HTML}{EB9113}
\definecolor{quarto-callout-tip-color}{HTML}{00A047}
\definecolor{quarto-callout-caution-color}{HTML}{FC5300}
\definecolor{quarto-callout-color-frame}{HTML}{acacac}
\definecolor{quarto-callout-note-color-frame}{HTML}{4582ec}
\definecolor{quarto-callout-important-color-frame}{HTML}{d9534f}
\definecolor{quarto-callout-warning-color-frame}{HTML}{f0ad4e}
\definecolor{quarto-callout-tip-color-frame}{HTML}{02b875}
\definecolor{quarto-callout-caution-color-frame}{HTML}{fd7e14}
\makeatother
\makeatletter
\@ifpackageloaded{caption}{}{\usepackage{caption}}
\AtBeginDocument{%
\ifdefined\contentsname
  \renewcommand*\contentsname{Table of contents}
\else
  \newcommand\contentsname{Table of contents}
\fi
\ifdefined\listfigurename
  \renewcommand*\listfigurename{List of Figures}
\else
  \newcommand\listfigurename{List of Figures}
\fi
\ifdefined\listtablename
  \renewcommand*\listtablename{List of Tables}
\else
  \newcommand\listtablename{List of Tables}
\fi
\ifdefined\figurename
  \renewcommand*\figurename{Figure}
\else
  \newcommand\figurename{Figure}
\fi
\ifdefined\tablename
  \renewcommand*\tablename{Table}
\else
  \newcommand\tablename{Table}
\fi
}
\@ifpackageloaded{float}{}{\usepackage{float}}
\floatstyle{ruled}
\@ifundefined{c@chapter}{\newfloat{codelisting}{h}{lop}}{\newfloat{codelisting}{h}{lop}[chapter]}
\floatname{codelisting}{Listing}
\newcommand*\listoflistings{\listof{codelisting}{List of Listings}}
\usepackage{amsthm}
\theoremstyle{definition}
\newtheorem{definition}{Definition}[section]
\theoremstyle{remark}
\AtBeginDocument{\renewcommand*{\proofname}{Proof}}
\newtheorem*{remark}{Remark}
\newtheorem*{solution}{Solution}
\newtheorem{refremark}{Remark}[section]
\newtheorem{refsolution}{Solution}[section]
\makeatother
\makeatletter
\makeatother
\makeatletter
\@ifpackageloaded{caption}{}{\usepackage{caption}}
\@ifpackageloaded{subcaption}{}{\usepackage{subcaption}}
\makeatother

\ifLuaTeX
  \usepackage{selnolig}  % disable illegal ligatures
\fi
\usepackage{bookmark}

\IfFileExists{xurl.sty}{\usepackage{xurl}}{} % add URL line breaks if available
\urlstyle{same} % disable monospaced font for URLs
\hypersetup{
  pdftitle={Systems of equations},
  colorlinks=true,
  linkcolor={blue},
  filecolor={Maroon},
  citecolor={Blue},
  urlcolor={Blue},
  pdfcreator={LaTeX via pandoc}}


\title{Systems of equations}
\author{}
\date{}

\begin{document}
\maketitle

\renewcommand*\contentsname{Table of contents}
{
\hypersetup{linkcolor=}
\setcounter{tocdepth}{3}
\tableofcontents
}

Lets review the method you use in high-school to solve systems of
equations - the elimination method - and then rewrite it under a new
notation - the matrix and vector notation.

A few examples will be given.

Key concepts: \emph{systems of equations in matrix-vector notation,
elimination, pivot, rank, conditions for 1,0 or infinite solutions.}

\subsection{Matrix-vector notation for a
system}\label{matrix-vector-notation-for-a-system}

Consider the system with two equations, called \(l_1\) and \(l_2\):

\begin{equation}\phantomsection\label{eq-sist}{
\begin{cases}
x-4y =2\\
2x-6y = 5
\end{cases}
}\end{equation}

In Equation~\ref{eq-sist} we find two equations and two unknowns \(x\)
and \(y\). We want their values such that both equations are satisfied.
(this system represents the interception of two lines)

Using the matrix-vector notation we can write Equation~\ref{eq-sist} as:

\begin{equation}\phantomsection\label{eq-matrix_sist}{
\begin{pmatrix} 1 & -4\\ 2 & -6 \end{pmatrix}\begin{pmatrix}x\\y\end{pmatrix}= \begin{pmatrix}2\\5\end{pmatrix}
}\end{equation}

Lets read Equation~\ref{eq-matrix_sist} in words: the \(2\) by \(2\)
matrix of coefficients is multiplied by the column vector
\(\begin{pmatrix}x\\y\end{pmatrix}\) , the result is
\(\begin{pmatrix}2\\5\end{pmatrix}\). This is \emph{one} equation with
\emph{one} unknown, the column vector
\(\begin{pmatrix}x\\y\end{pmatrix}\). Traditionally we write
Equation~\ref{eq-matrix_sist} as \(A \mathbf{x} =\mathbf{b}\).

How do we multiply a vector by a matrix?

\emph{Answer:}

\[
\overbrace{\begin{pmatrix} 1 & -4\\ 2 & -6 \end{pmatrix}\begin{pmatrix}x\\y\end{pmatrix}}^\text{matrix-vector mult.} =\overbrace{\begin{pmatrix}1\cdot x-4\cdot y \\ 2\cdot x-6\cdot y\end{pmatrix}}^\text{scalar mult.}
\]

Matrix times a vector on the lhs is just a super-compact way of writing
the vector on the rhs. Moreover, notice the shapes of the matrix and
vectors, this is very, very important. A \(2\) by \(2\) matrix times a
\(2\) by \(1\) column vector yields a \(2\) by \(1\) column vector! If
you understand this it should not be a problem to see what shapes are or
not compatible, check this:

\begin{itemize}
\item
  {[}\(2\times2\){]}{[} \(3\times 1\){]} \(=\) Nonsense
\item
  {[}\(3\times2\){]}{[} \(2\times 1\){]} \(=\) {[}\(3\times 1\){]}
\item
  {[}\(2\times3\){]}{[} \(3\times 1\){]} \(=\) {[}\(2\times 1\){]}
\item
  {[}\(3\times2\){]}{[} \(3\times 1\){]} \(=\) Nonsense
\item
  {[}\(1\times3\){]}{[} \(3\times 1\){]} \(=\) {[}\(1\times 1\){]}
\end{itemize}

\subsection{Solving the system using the Elimination
method}\label{solving-the-system-using-the-elimination-method}

Lets solve the system Equation~\ref{eq-sist} using the traditional rules
we already know from high-school, recall:

\begin{enumerate}
\def\labelenumi{\arabic{enumi}.}
\tightlist
\item
  you can replace an equation by itself times some constant.
\item
  you can replace any one equation by the sum both equations.
\item
  you can isolate \(x\) or \(y\) in one equation and substitute in the
  other equation.
\end{enumerate}

In other words, 1. and 2., just say this: you can replace \(l_1\) or
\(l_2\) by some convenient combination \(al_1+bl_2\). Rule 3. is known
as back-substitution.

Applying any one of these operations yields another and equivalent
system of equations.

The central idea of the Elimination method is use linear combination of
equations (1. and 2.) to eliminate variables and thus giving us an
\emph{equivalent} and \emph{easier to solve} system. To eliminate
variables we make clever use of rules \(1\) and \(2\). Once the system
is simple enough we can use rule \(3\). How do you know what is or not a
good combination? We'll see that with examples. But the guiding
principle is to use the pivots.

This recaps what you know, now lets use these rules to solve the
Equation~\ref{eq-sist} and in parallel see the corresponding
matrix-vector version.

\begin{itemize}
\item
  \textbf{step 1:} Replace equation \(l_2\) by, \(l_2\) minus twice the
  equation \(l_1\), i.e., make the new second equation \(l_2'\) into
  \(l_2-2l_1\). This gives us:

  \[
  \begin{cases}x-4y =2\\2x-6y = 5\end{cases} \overset{l_2'=l_2-2l_1}{\longrightarrow}\begin{cases} x-4y =2\\ 2y = 1\end{cases}
  \]

  Correspondingly we subtract from row \(l_2\) twice the row \(l_1\) in
  Equation~\ref{eq-matrix_sist}, giving us

  \[
  \begin{pmatrix} 1 & -4\\ 2 & -6 \end{pmatrix}\begin{pmatrix}x\\y\end{pmatrix}= \begin{pmatrix}2\\5\end{pmatrix}\overset{l_2'=l_2-2l_1}{\longrightarrow} \begin{pmatrix} 1 & -4\\ 0 & 2 \end{pmatrix}\begin{pmatrix}x\\y\end{pmatrix}= \begin{pmatrix}2\\1\end{pmatrix}
  \]

  A good way to look at this is to focus on using the \(1\), to
  eliminate the \(2\). This entry of the matrix we focus on is called a
  pivot entry!
\item
  \textbf{step 2:} Multiply equation \(l_2\) by \(1/2\):

  \[
  \begin{cases} x-4y =2\\ 2y = 1\end{cases}\overset{l_2'=1/2l_2}{\longrightarrow} \begin{cases} x-4y =2\\ y = 1/2\end{cases}
  \]

  In matrix-vector notation we find:

  \[
  \begin{pmatrix} 1 & -4\\ 0 & 2 \end{pmatrix}\begin{pmatrix}x\\y\end{pmatrix}= \begin{pmatrix}2\\1\end{pmatrix}\overset{l_2'=1/2l_2}{\longrightarrow} \begin{pmatrix} 1 & -4\\ 0 & 1 \end{pmatrix}\begin{pmatrix}x\\y\end{pmatrix}= \begin{pmatrix}2\\1/2\end{pmatrix}
  \]
\item
  \textbf{step 3:} Focusing on the second pivot entry, we eliminate the
  entry, \(-4\), by replacing \(l_1\) by \(l_1\) plus four times
  \(l_2\):

  \[
  \begin{cases} x -4y =2\\ y = 1/2\end{cases}\overset{l_1'=l_1+4l_2}{\longrightarrow}\begin{cases} x =4\\ y = 1/2\end{cases}
  \]

  In matrix-vector notation we find:

  \[
  \begin{pmatrix} 1 & -4\\ 0 & 1 \end{pmatrix}\begin{pmatrix}x\\y\end{pmatrix}= \begin{pmatrix}2\\1/2\end{pmatrix}\overset{l_1'=l_1+4l_2}{\longrightarrow} \begin{pmatrix} 1 & 0\\ 0 & 1 \end{pmatrix}\begin{pmatrix}x\\y\end{pmatrix}= \begin{pmatrix}4\\1/2\end{pmatrix}
  \]

  From which we can read the final result \(x=4\) and \(y=1/2\).
\end{itemize}

\textbf{Better notation:} Going through the three steps again we notice
we can improve our matrix-vector notation by suppressing from it the
column \(\begin{pmatrix}x\\y\end{pmatrix}\) and writing instead the
steps as:

\[
\left(\begin{matrix} 1 & -4 \\ 2 & -6 \end{matrix}\;\middle|\; \begin{matrix} 2\\5\end{matrix}\right)
\overset{l_2'=l_2-2l_1}{\longrightarrow} 
\left(\begin{matrix} 1 & -4 \\ 0 & 2 \end{matrix}\;\middle|\; \begin{matrix} 2\\1\end{matrix}\right)
\overset{l_2'=1/2l_2}{\longrightarrow}
\left(\begin{matrix} 1 & -4 \\ 0 & 1 \end{matrix}\;\middle|\; \begin{matrix} 2\\1/2\end{matrix}\right)
\overset{l_1'=l_1+4l_2}{\longrightarrow}
\left(\begin{matrix} 1 & 0 \\ 0 & 1 \end{matrix}\;\middle|\; \begin{matrix} 4\\1/2\end{matrix}\right)
\]

From now on we'll adopt this way of writing systems of equations, its
called the extended matrix notation, because we appended a new column to
right side of the \(2\times2\) matrix. From the extended matrix we read
the solution as follows

\[
\left(\begin{matrix} 1 & 0 \\ 0 & 1 \end{matrix}\;\middle|\; \begin{matrix} 4\\1/2\end{matrix}\right)\longrightarrow \begin{pmatrix} 1 & 0\\ 0 & 1 \end{pmatrix}\begin{pmatrix}x\\y\end{pmatrix}= \begin{pmatrix}4\\1/2\end{pmatrix}\longrightarrow \begin{pmatrix}x\\y\end{pmatrix}= \begin{pmatrix}4\\1/2\end{pmatrix}
\]

where in the last step we multiplied the vector by the matrix.

The vector

\[
\begin{pmatrix}4\\1/2\end{pmatrix}
\]

is the solution of

\[
\begin{pmatrix} 1 & -4\\ 2 & -6 \end{pmatrix}\begin{pmatrix}x\\y\end{pmatrix}= \begin{pmatrix}2\\5\end{pmatrix}
\]

Meaning, the solution of

\[
\begin{cases}x-4y =2\\2x-6y = 5\end{cases}
\]

is:

\[
\begin{cases}x =4\\y=1/2\end{cases}
\]

\subsection{More examples of the elimination
method}\label{more-examples-of-the-elimination-method}

The following examples will be given:

\begin{itemize}
\item
  A system with one solution
\item
  A system with no solution
\item
  A system with many solutions
\item
  Another system with many solutions
\end{itemize}

\subsubsection{Example 1: A system with one
solution}\label{example-1-a-system-with-one-solution}

Consider the system, which we write in three different notations.

\begin{equation}\phantomsection\label{eq-possible_1}{
\begin{cases}
x+y-z=1\\
2x-y+2z = 9\\
2y=-x+z
\end{cases}
\leftrightsquigarrow
\begin{pmatrix}
1 & 1 & -1\\
2 & -1 & 2\\
1 & 2 &-1
\end{pmatrix}
\begin{pmatrix}
x\\
y\\
z
\end{pmatrix}
=
\begin{pmatrix}
1\\
9\\
0
\end{pmatrix}
\leftrightsquigarrow
\left(
\begin{matrix}
1 & 1 & -1\\
2 & -1 & 2\\
1 & 2 & -1
\end{matrix}
\;\middle|\;
\begin{matrix}
1\\
9\\
0
\end{matrix}
\right)
}\end{equation}

(This system represents the interception of three planes at one point,
each row of the matrix is a vector perpendicular to a plane)

Again we'll use the extended notation during the elimination algorithm,
because we don't want to carry around the \(x\), \(y\) and \(z\) at each
step.

\[
\begin{align}
&\left(\begin{matrix}1 & 1 & -1\\2 & -1 & 2 \\1 & 2 & -1 \end{matrix}\;\middle|\;\begin{matrix}1\\9\\0 \end{matrix}\right)
 \overset{l_2'=l_2-2l_1\\l_3'=l_3-l_1}{\longrightarrow}
\left(\begin{matrix}1 & 1 & -1 \\0 & -3 & 4 \\0 & 1 & 0 \end{matrix}\;\middle|\;\begin{matrix}1\\7\\-1\end{matrix}\right) 
\overset{l_2 \leftrightarrow l_3}{\longrightarrow}
\left(\begin{matrix}1 & 1 & -1 \\0 & 1 & 0\\0 & -3 & 4\end{matrix}\;\middle|\;\begin{matrix}1\\-1\\7\end{matrix}\right)\\
\overset{l_3'=l_3+3l_2}{\longrightarrow}
&\left(\begin{matrix}1 & 1 & -1 \\0 & 1 & 0 \\0 & 0 & 4 \end{matrix}\;\middle|\;\begin{matrix}1\\-1\\4\end{matrix}\right) 
\overset{l_3'=1/4l_3}{\longrightarrow}\left(\begin{matrix}1 & 1 & -1 \\0 & 1 & 0 \\0 & 0 & 1 \end{matrix}\;\middle|\;\begin{matrix}1\\-1\\1\end{matrix}\right) 
\overset{l_1'=l_1+l_3}{\longrightarrow}\left(\begin{matrix}1 & 1 & 0 \\0 & 1 & 0 \\0 & 0 & 1 \end{matrix}\;\middle|\;\begin{matrix}2\\-1\\1\end{matrix}\right) \\
\overset{l_1'=l_1-l_2}{\longrightarrow}
&\left(\begin{matrix}1 & 0 & 0 \\0 & 1 & 0 \\0 & 0 & 1 \end{matrix} \;\middle|\;\begin{matrix}3\\-1\\1\end{matrix}\right)
\end{align}
\]

Each step has its own commentary:

\begin{enumerate}
\def\labelenumi{\arabic{enumi}.}
\item
  On the first step we use the pivot \(A_{11}=1\) to eliminate the
  entries \(2\) and \(1\) bellow.
\item
  The second step we switched equations, so as to bring a pivot \(1\) at
  position \(A_{22}\).
\item
  The third step consists in using this pivot to eliminate the entry
  \(-3\) below.
\item
  In step four we divided \(l_3\) by \(4\) so that that the pivot
  \(A_{33}\) is \(1\) instead of \(4\).
\item
  At step five we use \(A_{33}\) to eliminate the entry \(-1\) above it.
\item
  Step six we just used the pivot \(A_{22}\), to eliminate the entry
  \(1\) at \(A_{12}\).
\end{enumerate}

After simplification, it is time to go back to the original notation -
we find:

\[
\begin{cases}
x = 3\\y=-1\\z=1
\end{cases}
\]

which is the solution of the system of equations! Correspondingly, the
solution for the matrix-vector notation is the following vector

\[
\begin{pmatrix}
2\\
-1\\
1
\end{pmatrix}
\]

\emph{Important observations (that we'll come very useful later):} the
form of the matrix after all these l.c. of rows, it has the form:

\[
\begin{pmatrix}I\end{pmatrix}
\]

and its rank is \(3\), meaning, the number of pivots if \(3\). Note as
well that if we consider each column of the matrix as a vector, then we
find \(3\) independent vectors. We write \(r=3\).

\begin{definition}[]\protect\hypertarget{def-rank}{}\label{def-rank}

{[}rank \(r\) of a matrix \(A\){]} = \(r\) = {[}\# of pivots{]} ={[}\#
of indep columns of \(A\){]} = {[}\# of indep rows of \(A\){]}

\end{definition}

Moreover, notice that the rank of the extended matrix \(r^*\) is also
\(3\). And that the number of columns \(n=3\) as well.

From this example we see something that happens in general:

\emph{A system of equations} \(A\mathbf{x}=\mathbf{b}\) \emph{have one
solution provided} \(r=r^*=n\).

\subsubsection{Example 2: A system with no
solution}\label{example-2-a-system-with-no-solution}

The system this time is:

\begin{equation}\phantomsection\label{eq-impossible_1}{
\begin{cases}
2x-y=8\\
y+2x = 4\\
x=-y-1
\end{cases}
\leftrightsquigarrow
\begin{pmatrix}
2 & -1\\
2 & 1\\
1 & 1
\end{pmatrix}
\begin{pmatrix}
x\\
y
\end{pmatrix}
=
\begin{pmatrix}
8\\
4\\
-1
\end{pmatrix}
\leftrightsquigarrow
\left(
\begin{matrix}
2 & -1 \\
2 & 1  \\
1 & 1 
\end{matrix}
\;\middle|\;
\begin{matrix}
8\\
4\\
-1
\end{matrix}
\right)
}\end{equation}

To find how many solutions \(\begin{pmatrix}x\\y\end{pmatrix}\) are
there, we try to compute them using the Elimination algorithm:

\[
\begin{align}
&\left(\begin{matrix}2 & -1 \\2 & 1 \\1 & 1  \end{matrix}\;\middle|\;\begin{matrix}8\\4\\-1\end{matrix}\right)
\overset{l_1\leftrightarrow l_3}{\longrightarrow}
\left(\begin{matrix}1 & 1\\ 2 & 1 \\2 & -1 \end{matrix} \;\middle|\;\begin{matrix}-1\\4\\8\end{matrix}\right)
\overset{l_2'=l_2-2l_1\\l_3'=l_3-2l_1}{\longrightarrow}
\left(\begin{matrix}1 & 1 \\0 & -1 \\0 & -3 \end{matrix}\;\middle|\;\begin{matrix}-1\\6\\10\end{matrix}\right)\\
\overset{l_2'=-l_2\\l_3'=l_3-3l_2}{\longrightarrow}
&\left(\begin{matrix}1 & 1 \\0 & 1 \\0 & 0 \end{matrix} \;\middle|\; \begin{matrix}-1\\-6\\-8\end{matrix}\right)
\overset{l_1'=l_1-l_2}{\longrightarrow}
\left(\begin{matrix}1 & 0 \\0 & 1 \\0 & 0 \end{matrix} \;\middle|\;\begin{matrix}5\\-6\\-8\end{matrix}\right)
\end{align}
\]

This means:

\[
\left(\begin{matrix}1 & 0 \\0 & 1 \\0 & 0 \end{matrix} \;\middle|\;\begin{matrix}5\\-6\\-8\end{matrix}\right)
\leftrightsquigarrow
\begin{pmatrix}1 & 0 \\0 & 1 \\0 & 0 \end{pmatrix} \begin{pmatrix}x\\y\end{pmatrix}=\begin{pmatrix}5\\-6\\-8\end{pmatrix}
\]

Now that most simplification is done, lets convert back to the system's
notation:

\begin{equation}\phantomsection\label{eq-impossible_2}{
\begin{cases}
x=5\\
y=-6\\
0=-8
\end{cases}
}\end{equation}

It is clearly impossible. No choice of \(x\) or \(y\) makes this three
statements true simultaneously! Since Equation~\ref{eq-impossible_1} is
equivalent to Equation~\ref{eq-impossible_2}, therefore our original
system Equation~\ref{eq-impossible_1} has no solution as well.

Observations: Notice the form of the matrix is

\[
\begin{pmatrix}I\\\mathbf{0}\end{pmatrix}
\]

the fact that \(r=2\), \(r^*=3\) and \(n=2\). We see in this example
something which happens in general and which we'll justify later:

\emph{A system of equations} \(A\mathbf{x}=\mathbf{b}\) \emph{have no
solution provided} \(r<r^*\).

\subsubsection{Example 3: A system with many
solutions}\label{example-3-a-system-with-many-solutions}

The system is:

\begin{equation}\phantomsection\label{eq-many_1}{
\begin{cases}
x+y-z=0\\
2x-y+2z = 0\\
\end{cases}
\leftrightsquigarrow
\begin{pmatrix}
1 & 1 & -1\\
2 & -1 & 2\\
\end{pmatrix}
\begin{pmatrix}
x\\
y\\
z
\end{pmatrix}
=
\begin{pmatrix}
0\\
0
\end{pmatrix}
\leftrightsquigarrow
\left(
\begin{matrix}
1 & 1 & -1\\
2 & -1 & 2 \\
\end{matrix}
\;\middle|\;
\begin{matrix}
0\\0
\end{matrix}
\right)
}\end{equation}

Solving:

\[
\begin{align}
&\begin{pmatrix}1 & 1 & -1 &\bigm| & 0\\2 & -1 & 2 &\bigm| & 0\\\end{pmatrix}
\overset{l_2'=l_2-2l_1}{\longrightarrow}
\begin{pmatrix}1 & 1 & -1 &\bigm| & 0\\0 & -3 & 4 &\bigm| & 0\end{pmatrix} 
\overset{l_2'=-1/3l_2}{\longrightarrow}
\begin{pmatrix}1 & 1 & -1 &\bigm| & 0\\0 & 1 & -4/3 &\bigm| & 0\end{pmatrix}\\
\overset{l_1'=l_1-l_2}{\longrightarrow}
&\begin{pmatrix}1 & 0 & 1/3 &\bigm| & 0\\0 & 1 & -4/3 &\bigm| & 0\end{pmatrix} 
\end{align}
\]

Once again, this notation means:

\begin{equation}\phantomsection\label{eq-ex4}{
\begin{pmatrix}1 & 0 & 1/3 &\bigm| & 0\\0 & 1 & -4/3 &\bigm| & 0\end{pmatrix} 
\leftrightsquigarrow 
\begin{pmatrix}1 & 0 & 1/3\\0 & 1 & -4/3 \end{pmatrix}
\begin{pmatrix}x\\y\\z\end{pmatrix}
=
\begin{pmatrix}0\\0\end{pmatrix}
}\end{equation}

Converting back to the system notation have:

\begin{equation}\phantomsection\label{eq-many_2}{
\begin{cases}
x+1/3z =0\\
y-4/3z=0
\end{cases} 
\iff
\begin{cases}
x=-1/3z\\
y=4/3z
\end{cases}
}\end{equation}

Asking what \((x,y,z)\in \mathbb{R}^3\) that satisfy the equation
Equation~\ref{eq-many_1} is equivalent to ask, what is
\(x,y,z\in \mathbb{R}\) that satisfy Equation~\ref{eq-many_2} . Each
real \(z\) we choose gives us the corresponding \(x\) and \(y\); as a
consequence we have many solution. In other words, the solution is

\begin{equation}\phantomsection\label{eq-parameter_dep_sol}{
\begin{pmatrix}
x\\
y\\
z
\end{pmatrix}
=
\begin{pmatrix}
-1/3z\\
4/3z\\
z
\end{pmatrix}
}\end{equation}

Observations: The aspect of the final matrix is:

\[
\begin{pmatrix}I \,\,F\end{pmatrix}
\]

where the \(F\) block is the third column \((1/3, -4/3)^\intercal\),
while \(I\) is the \(2\) by \(2\) identity.

Additionally, \(r=2\), \(r^*=2\) and \(n=3\). A general rule (to be
explained later) is:

\emph{A system of equations} \(A\mathbf{x}=\mathbf{b}\) \emph{have
infinite solutions provided} \(r=r^*<n\).

\subsubsection{Example 4: Another system with many
solutions}\label{example-4-another-system-with-many-solutions}

What is the solution \((x,y, z, w)\) for the following system:

\[
\begin{cases}
&x&+ &2 y&+&2 z &+&2w &= 1\\
&2x&+&4y&+&6z&+&8w &= 2\\
&3x&+&6y&+&8z&+&10w &=3
\end{cases}
\]

Using elimination algorithm we make linear combinations of the equation
with the goal of eliminating variables, here is one way to go\\
\begin{equation}\phantomsection\label{eq-example_4_system}{
\begin{align}
&\begin{pmatrix}
1 & 2 & 2 & 2 &\bigm|1\\
2 & 4 & 6 & 8 &\bigm|2\\
3 & 6 & 8 & 10 &\bigm| 3
\end{pmatrix}
\overset{l_2' = l_2-2l_1}{\longrightarrow}
\begin{pmatrix}
1 & 2 & 2 & 2 &\bigm| 1\\
0 & 0 & 2 & 4 &\bigm| 0\\
3 & 6 & 8 & 10 &\bigm| 3
\end{pmatrix}
\overset{l_3' = l_3-3l_1}{\longrightarrow}
\begin{pmatrix}
1 & 2 & 2 & 2 &\bigm| 1\\
0 & 0 & 2 & 4 &\bigm| 0\\
0 & 0 & 2 & 4 &\bigm| 0
\end{pmatrix}\\
&\overset{l_3'=l_3-l_2}{\longrightarrow}
\begin{pmatrix}
1 & 2 & 2 & 2 &\bigm| 1\\
0 & 0 & 2 & 4 &\bigm| 0\\
0 & 0 & 0 & 0 &\bigm| 0
\end{pmatrix}
\overset{l_2'=1/2l_2}{\longrightarrow}
\begin{pmatrix}
1 & 2 & 2 & 2 &\bigm| 1\\
0 & 0 & 1 & 2 &\bigm| 0\\
0 & 0 & 0 & 0 &\bigm| 0
\end{pmatrix}
\overset{l_1'=l_1-2l_2}{\longrightarrow}
\begin{pmatrix}
1 & 2 & 0 & -2 &\bigm| 1\\
0 & 0 & 1 & 2 &\bigm| 0\\
0 & 0 & 0 & 0 &\bigm| 0
\end{pmatrix}
\end{align}
}\end{equation}

We simplified it as much as we can, going back to the system's notation
we have:

\[
\begin{cases}
x + 2 y & &- &2w &=1\\
&z &+ &2w &=0
\end{cases}
\]

Now, promote \(y\) and \(w\) into parameters and express \(x\) and \(z\)
in term of them (if why we take this step is not natural, it will be
soon)

\begin{equation}\phantomsection\label{eq-sol_1}{
\begin{cases}
x + 2 y & &- &2w &=1\\
&z &+ &2w &=0
\end{cases}
\longrightarrow
\begin{cases}
x &=1-2y&-&2w\\
z &=&-&2w
\end{cases}
}\end{equation}

For each values we assign to \(y\) and \(w\) we get one solution!
There's an infinite number of them.

This example is given so that you can see, that the final aesthetic of
the simplified \(A\) may have the \(I\) and \(F\) blocks mixed!

This is also a matrix with form:

\[
\begin{pmatrix}I \,\,F\end{pmatrix}
\]

The rank is \(r=2\), the rank of the extended matrix is \(r^*=2\),
meanwhile \(n=4\). Since \(r=r^*<n\) we have an infinite number of
solutions.

\subsubsection{Example 5: A system that may have one, none or many
solutions}\label{example-5-a-system-that-may-have-one-none-or-many-solutions}

What is the solution of:

\[
\begin{cases}
&x&+ &k y&+&2 z &= 1\\
&x&+&y&+&(k+1)z &= k\\
&-x&-&y&-&z &=k+1
\end{cases}
\]

Using the extended matrix formulation we use the pivots to eliminate
entries:

\[ \begin{align} &\begin{pmatrix} 1 & k & 2  &\bigm|1\\ 1 & 1 & k+1 &\bigm|k\\ -1 & -1 & -1 &\bigm| k+1 \end{pmatrix} \overset{l_2' = l_2-l_1\\l_3'=l_3+l_1}{\longrightarrow} \begin{pmatrix} 1 & k & 2 &\bigm| 1\\ 0 & 1-k & k-1 &\bigm| k-1\\ 0 & k-1 & 1 &\bigm| k+2 \end{pmatrix} \overset{l_3' = l_3+l_2}{\longrightarrow} \begin{pmatrix} 1 & k & 2  &\bigm| 1\\ 0 & 1-k & k-1 &\bigm| k-1\\ 0 & 0 & k &\bigm| 2k+1 \end{pmatrix} \end{align} \]

Now, depending upon on the value \(k\):

For \(k=1\) then

\begin{equation}\phantomsection\label{eq-k1}{
\begin{pmatrix} 1 & 1 & 2  &\bigm|1\\ 0 & 0 & 0 &\bigm|0\\ 0 & 0 & 1 &\bigm| 3 \end{pmatrix} \overset{l_2 \leftrightarrow l_3}{\longrightarrow} \begin{pmatrix} 1 & 1 & 2  &\bigm|1\\ 0 & 0 & 1 &\bigm|3\\ 0 & 0 & 0 &\bigm| 0 \end{pmatrix}\overset{l_1'=l_1-2l_2 }{\longrightarrow}\begin{pmatrix} 1 & 1 & 0  &\bigm|-5\\ 0 & 0 & 1 &\bigm|3\\ 0 & 0 & 0 &\bigm| 0 \end{pmatrix}
}\end{equation}

which has the form:

\[
\begin{pmatrix}I\,\,F \end{pmatrix}
\]

also: \(r=2\), \(r^*=2\) and \(n=3\).

Since \(r=r^*<n\) we conclude we have an infinite number of solutions as
we readily check by actually computing them. From Equation~\ref{eq-k1}
we know:

\[
\begin{cases}
x + y = -5\\
z=3
\end{cases}
\]

Since we have two equations and three unknowns, we promote one of them,
let it be \(x\), to the status of a parameter \(x_0\), hence:

\[
\begin{pmatrix}x\\y\\z\end{pmatrix}=\begin{pmatrix}x_0\\-5-x_0\\3\end{pmatrix}=\begin{pmatrix}0\\-5\\3\end{pmatrix}+x_0\begin{pmatrix}1\\-1\\0\end{pmatrix}
\]

For each \(x_0\) of our choice we have a distinct solution. There is
infinite of them.

For \(k=0\)

\[
\begin{pmatrix} 1 & 0 & 2  &\bigm|1\\ 0 & 1 & -1 &\bigm|-1\\ 0 & 0 & 0 &\bigm| 1 \end{pmatrix} 
\]

The system has the form:

\[
\begin{pmatrix}
I & F\\
\mathbf{0} &\mathbf{0}
\end{pmatrix}
\]

And note: \(r=2\), \(r^*=3\) and \(n=3\), thus \(r<r^*=n\) which tells
us there is no solution. In fact we can see why by rolling back to:

\[
\begin{cases}
x+2z=1\\
y-z =-1\\
0=1
\end{cases}
\]

There are no \(x\), \(y\) and \(z\) that satisfies these three equations
simultaneously.

If \(k\) is not \(0,1\) then we always have \(r=r^*=n\), thus, there is
always a unique solution.

\textbf{In summary:}

\begin{itemize}
\item
  \emph{A system of equations} \(A\mathbf{x}=\mathbf{b}\) \emph{have
  infinite solutions provided} \(r=r^*<n\).
\item
  \emph{A system of equations} \(A\mathbf{x}=\mathbf{b}\) \emph{have one
  solution provided} \(r=r^*=n\).
\item
  \emph{A system of equations} \(A\mathbf{x}=\mathbf{b}\) \emph{have no
  solutions provided} \(r<r^*\).
\end{itemize}

\begin{tcolorbox}[enhanced jigsaw, colback=white, breakable, toptitle=1mm, left=2mm, bottomtitle=1mm, opacitybacktitle=0.6, rightrule=.15mm, title=\textcolor{quarto-callout-note-color}{\faInfo}\hspace{0.5em}{Commentaries}, arc=.35mm, toprule=.15mm, opacityback=0, colframe=quarto-callout-note-color-frame, colbacktitle=quarto-callout-note-color!10!white, leftrule=.75mm, titlerule=0mm, coltitle=black, bottomrule=.15mm]

\begin{itemize}
\item
  For the moment assume this summary as a fact of life, later we'll
  justify in detail the meaning of this triple equality, for know just
  focus on learning elimination and testing the triple equality
\item
  How do we know how many pivots are there? Using l.c. of rows, simplify
  the matrix by eliminating as many entries as possible using the
  pivots.
\item
  A matrix where all entries below the pivots were killed is in
  triangular form. Back substitution can already be used at this stage.
  (we did not do this in the examples, but we could) If we proceed and
  also kill all entries above the pivots, we get the best matrix (this
  is what we did in the examples). A matrix is then said to be in
  reduced row echelon form. Back substitution can also used at this
  final stage.
\item
  With these three statements we can decide whether a system has one,
  none or infinite solution. But thinking about how did we get \(r\) and
  \(r^*\)? We did use elimination, which is a labor intensive process,
  to bring \(A\mathbf{x}=\mathbf{b}\) into a simplified form. So much
  work was done, one might as well solve the entire thing we from the
  solution itself evaluate whether there is one, none or infinite
  solutions. Here comes a very important point, these three statements
  are worthless as a practical point of view to decide in which case are
  we. These three are like a tip of an iceberg, and what's underneath is
  beautiful way to understand why some systems have one, none or
  infinite solution. For the moment, we use them as facts of life, later
  we uncover what is going on. These statements are only useful in
  situations where the matrix depend on some parameter \(k\) and we
  which, after putting the matrix in triangular form, we want to check
  for which values there is one, none or infinite solutions.
\end{itemize}

\end{tcolorbox}

\section{Exercises}\label{exercises}

solve 1.3 \textgreater{} 1

solve 1.3 \textgreater{} 2 \textgreater{} b (In this exercise, the
matrices are already in triangular form, just use back subst)

solve 1.3 \textgreater{} 3 (In this exercise we have to use elimination
to get the triangle form and them use back subst)

solve 1.3 \textgreater{} 4 and 6 (skip! This exercise could be done
using the concepts of section linear (in)dependence, but it is much
easier to relate that section with solving Ax=0, because then, I can use
elimination to simplify the cols and then see the dependencies. If I
just use the concepts from the linear (in)dependence section will be
harder, because, elimination was not introduced yet and neither was the
connection of the (in)dependence concept made with the problem of
solving Ax=0)

solve 1.3 \textgreater{} 5

solve 1.3 \textgreater{} 7 \textgreater{} a

solve 1.3 \textgreater{} 8 \textgreater{} d

\[
\left(  \begin{matrix}    1 & 2 & 3 \\    1 & 2 & 3 \\    1 & 2 & 3 \\    1 & 2 & 3 \\  \end{matrix}  \;\middle|\;       \begin{matrix}      4  \\      4  \\      4  \\      4  \\    \end{matrix}  \right)
\]




\end{document}
